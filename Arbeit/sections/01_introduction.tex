\chapter{Einleitung}\label{chap:einleitung}

\section{Motivation}\label{sec:motivation}

Software-Systeme werden zunehmend komplexer: Damit einhergeht eine Zunahme möglicher Fehler bei der Nutzung -- mit enormen Auswirkungen, wie eine Studie des Consortium for Information \& Software Quality (CISQ) \cite{krasner2021cost} aufzeigt: Für das Jahr 2020 schätzte das CISQ die Kosten unentdeckter Softwarefehler für die US-amerikanische Wirtschaft auf Basis öffentlicher Nachrichten und Informationen auf 1,56 Billionen US-Dollar, 2018 lag dieser Wert noch bei 1,28 Billionen US-Dollar.

Im Zuge dessen rückt im Bereich der Software-Entwicklung vermehrt die Bedeutung adäquater Methoden des Softwaretestens in den Vordergrund, mit dem Ziel mögliche Fehler frühzeitig zu erkennen und zu eliminieren. Dabei wurde in der Vergangenheit in vielen Fällen auf die Erfahrung und Expertise von Software-Entwickler*innen vertraut, systematische Verfahren zur Generierung von Testfällen fanden nur wenig Beachtung. Als eine von verschiedenen systematischen Methoden zur Testfallgenerierung etabliert sich das Prinzip des Combinatorial Testing: Basierend auf dem Ansatz, dass die Interaktion einiger, weniger Parameter für einen Großteil aller Softwarefehler verantwortlich ist, ermöglicht Combinatorial Testing die effektive und kostengünstige Erstellung von Testbeständen und bietet zudem die Möglichkeit, die Güte jener Testbestände auf mathematisch-kombinatorischer Ebene zu quantifizieren.

\section{Ziel der Arbeit}\label{sec:zielderArbeit}

Ziel dieser Arbeit ist es, die Grundlagen des Softwaretestens und die Theorie des Combinatorial Testings darzulegen und basierend darauf am Beispiel eines Anwendungsfalls aus dem Bereich der aktuariellen Software die praktische Umsetzung von Combinatorial Testing auszuprobieren. Anhand verschiedener Teilaspekte soll aufgezeigt werden, wie Combinatorial Testing die Qualitätssicherung, im Speziellen die Generierung von adäquaten Testfällen, optimieren kann.

Als Anwendungsbeispiel dient die Beratungs- und Kalkulationsplattform für Lebensversicherungsprodukte der ALH-Gruppe \glqq Easy Web Leben\grqq{} \cite{easy_web}, welche zur Beantwortung folgender, übergeordneten Fragestellung dieser Arbeit genutzt wird: Können Testfälle bei komplexen Systemen wie Easy Web effizient mittels kombinatorischer Methoden erzeugt werden und welche Vorteile ergeben sich bei der Nutzung der Prinzipien des Combinatorial Testing? 

Unter anderem soll im Rahmen dieser Arbeit auch untersucht werden, welche Algorithmen und Tools zur Erstellung von Mengen an Testfällen existieren und in einem empirischen Vergleich die Nutzbarkeit verschiedener Algorithmen und Tools geprüft werden. 

\section{Verwandte Arbeiten}\label{sec:verwandteArbeiten}

Combinatorial Testing hat sich in den vergangenen Jahren zu einer verbreitete Methode zum Testen verschiedener Eingabe- und Konfigurationsmöglichkeiten von Software entwickelt: Verschiedene Algorithmen und Tools wurden in der Vergangenheit vorgestellt, die eine möglichst schnelle und effiziente Nutzung der Methoden des Combinatorial Testing ermöglichen. Dies bewirkt insbesondere, dass die Nutzung der Methoden des Combinatorial Testing auch in Bereichen der Forschung und Industrie weit verbreitet ist: Verschiedenste Forschungsarbeiten wie \cite{li2016applying,hagar2015introducing,smith2019measuring,ozcan2017applications,dhadyalla2014combinatorial,raunak2017combinatorial} untersuchten für unterschiedlichste Anwendungsfälle die Nutzbarkeit und Vorteile von Combinatorial Testing. Über alle Forschungsarbeiten hinweg zeigt sich, dass Combinatorial Testing sowohl im Bereich der Wissenschaft als auch in der Industrie einen Mehrwert bietet und zunehmend an Relevanz gewinnt.

Im Kontext der Finanz- und Versicherungswirtschaft begrenzen sich die Erkenntnisse im Wesentlichen auf die Arbeit von Mehta und Philip \cite{mehta2013applications}, in welcher die Anwendung von Combinatorial Testing am Beispiel verschiedener Softwarekomponenten des ehemaligen IT-Beratungsunternehmens iGATE (übernommen durch Capgemini) vorgestellt wird. Mehta und Philip konnten herausarbeiten, dass die Anwendung der Methoden des Combinatorial Testing die Anzahl benötigter Testfälle in erheblichem Maße reduziere, zugleich jedoch die Anwendung ein hohes Verständnis über die mathematisch-kombinatorischen Grundlagen erfordere und die richtige Wahl der Faktoren und ihrer Werte eine entscheidende Rolle spielten. 

Die Untersuchungen dieser Arbeit fügen sich der aufgeführten Forschung an und sollen insbesondere den Forschungsstand in Bezug auf die Anwendung im Kontext von aktuarieller Software erweitern. Darüber hinaus existieren in der Forschung keine empirischen Vergleiche aktuell verfügbarer Tools und Algorithmen in Bezug auf die Nutzung in einem praxisrelevanten Anwendungsfall.

\section{Struktur der Arbeit}\label{sec:strukturderArbeit}

Die Ausführungen dieser Arbeit sind folgendermaßen aufgebaut: Zunächst werden in \autoref{chap:theorie} die theoretische Grundlagen des Softwaretestens (vgl. \autoref{sec:einführungTest}) und der Methode des Combinatorial Testing (vgl. \autoref{sec:combinatorialTesting}) erläutert. Der Abschnitt zum Combinatorial Testing beinhaltet unter anderem auch die Vorstellung verschiedener Algorithmen und Tools zur Generierung von Mengen an Testfällen. Anschließend folgt in \autoref{chap:anwendungsfall} die Vorstellung des Anwendungsfalls mit detaillierten Ausführungen zum Testobjekt \glqq Easy Web Leben\grqq{} und relevanten versicherungstechnischen Grundlagen (vgl. \autoref{sec:testobjekt}). Im Anschluss daran werden verschiedene Teilaspekte zur Beantwortung der grundlegenden Fragestellungen aufgegriffen und deren Implementierung (vgl. \autoref{sec:implementierung}) und Ergebnisse vorgestellt (vgl. \autoref{sec:results}). Abschließend werden die Erkenntnisse der Resultate der untersuchten Teilfragestellungen in \autoref{chap:diskussion} diskutiert und im Kontext der übergeordneten, grundlegenden Fragestellung dieser Arbeit eingestuft. Mögliche Limitierungen und Optionen für zukünftige Forschungsarbeiten werden ebenfalls aufgeführt.
