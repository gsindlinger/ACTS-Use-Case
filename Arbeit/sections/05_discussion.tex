\chapter{Diskussion}\label{chap:diskussion}

Die Ergebnisse aller verschiedenen Teilfragestellungen zeigen auf, dass mittels Combinatorial Testing bei komplexen Systeen wie Easy Web Testfallmengen auf sinnvolle Art und Weise erstellt werden können. Dies gilt unabhängig vom konkreten Ansatz und des verwendeten Algorithmus beziehungsweise des verwendeten Tools: Alle verschiedenen Methoden führen in geringem Zeitaufwand zu geeigneten Ergebnissen im Sinne einer hohen praxisrelevanten Nutzbarkeit.

Diese grundlegende Erkenntnis spiegelt jedoch nur ein grobes Gesamtbild der Ergebnisse der verschiedenen Untersuchungen dieser Arbeit wider: Aus diesem Grund sollen im Folgenden die einzelnen Resultate der Teilfragestellungen aus \autoref{sec:implementierung} und \autoref{sec:results} zusammengefasst und interpretiert werden.

\begin{enumerate}
\item \textit{Ist es überhaupt möglich Testfälle für ein System wie Easy Web mittels Combinatorial Testing zu erstellen?}

ACTS (vgl. \autoref{subsub:acts}) erstellt via IPOG-Algorithmus (vgl. \autoref{subsub:ipog}) Testmengen zur vollständigen $t$-fachen Kombinationsabdeckung  für das Basis-System in weniger als 0,14 Sekunden (vgl. \autoref{subsec:ImplBasisModell}) und für das Erweiterte System (vgl. \autoref{subsec:ImplErweitertesModell}) in weniger als 0,92 Sekunden. Somit zeigt sich, dass sich in geringem zeitlichen Rahmen Testfälle für das Anwendungsbeispiel Easy Web erstellen lassen.

Dabei gilt es im Besonderen zu berücksichtigen, dass innerhalb des Anwendungsfalls Bedingungen integriert sind, die bei der Erstellung der Testmengen berücksichtigt werden mussten. Auch dies scheint angesichts der Ergebnisse des Basis-Systems (vgl. \autoref{subsec:resultsBasisModell}) und der Ergebnisse des Erweiterten Systems (vgl. \autoref{subsec:resultsComplexSystem}) keine nennenswerten Auswirkungen auf die Anwendbarkeit der Methoden des Combinatorial Testing in der Praxis zu besitzen.

Vielmehr lässt sich beobachten, dass weniger die Ausführung der Algorithmen selbst bei der Erstellung von Testmengen zur $t$-fachen Kombinationsabdeckung eine Herausforderung darstellen, sondern eher die vorgelagerten Schritte zur Ermittlung der Parameter, deren Ausprägungen und etwaigen Bedingungen, welche das Testsystem erfüllen sollte.

Die Ausführungen zur Implementierung des Basis-Systems (vgl. \autoref{subsec:ImplBasisModell}) und die des Erweiterten Systems (vgl. \autoref{subsec:ImplErweitertesModell}) zeigen die Relevanz einer adäquaten Analyse des Testobjekts: 
Die Kombinationsmöglichkeiten der verschiedenen Parameter, sowie unterschiedliche, zu testende Wertebereiche und Grenzwerte erfordern ein hohes Verständnis über Easy Web. Anschaulich lässt sich dies an den komplexen Strukturen des Fehlerverhaltens von Easy Web in \autoref{tab:beitragsgrenzen} erkennen: Eine unzureichende Analyse an dieser Stelle würde zu einer ungeeigneten Wahl der Werte der verschiedenen Parameter führen (vgl. \autoref{tab:äquivalenzklassenSimple} für das Basis-System, \autoref{tab:äquivalenzklassenComplex} für das Erweiterte System) und somit in ungeeigneten Testfällen resultieren. Ähnliche Beobachtungen konnten Mehta und Philip \cite{mehta2013applications} in ihrer Arbeit zur Anwendung von Combinatorial Testing im Kontext der Finanz- und Versicherungsbranche machen.


Erfahrungswerte in Bezug auf die Erstellung von Testfällen (vgl. \autoref{subsub:erfahrungsbasiertesTesten}) und die geeignete Nutzung systematischer Verfahren wie die Äquivalenzklassenmethode (vgl. \autoref{subsub:äquivalenklassenmethode}) und die Grenzwertanalyse (vgl. \autoref{subsub:grenzwertanalyse}) erwiesen sich im Rahmen dieser Arbeit zur Meisterung der beschriebenen Herausforderung als nützlich.

\item \textit{Schafft Combinatorial Testing Vorteile gegenüber einem zufälligen Testverfahren?}

Im Kontext der Resultate des Ansatzes des Random Testing zeigt sich, inwiefern die Nutzung der Methoden des Combinatorial Testing Vorteile gegenüber anderen, naiven Vorgehensweisen besitzt: Wie \autoref{subsec:resultsRandomTesting} aufzeigt, besitzt Random Testing eine geringe Effizienz in der Ausführung von 2\% und damit verbunden eine lange Ausführungszeit (vgl. \autoref{subsec:resultsRandomTesting}). Besonders für eine hohe Anzahl der zu erzeugenden Testfälle ($N=200,300$) macht sich dieser Effekt besonders bemerkbar. 

Hauptursache dafür sind die verschiedenen Bedingungen des Basis-Systems, welche in jeder Iteration eines zufällig generierten Testfalls geprüft werden müssen und in vielen Fällen nicht mit dem Testfall konsistent sind. Darüber hinaus fallen die Werte der Metriken des Combinatorial Testing ($t$-fache Kombinationsabdeckung, Variablen-Wert-Abdeckung und (0,75-$t$)-Vollständigkeit) bei den die zufällig generierten Testfallmengen gering aus. 

Der Ansatz des Combinatorial Testing ermöglicht im Gegensatz zum Random Testing die Erstellung von Testmengen, die wesentlich kleiner ausfallen und zugleich eine höhere, garantierte Qualitätsgüte im Sinne der Metriken des Combinatorial Testing (vgl. \autoref{subsec:masse}) besitzen. Außerdem sind die Algorithmen und Tools des Combinatorial Testing im Kontext der verschiedenen Bedingungen des Basis-Systems wesentlich schneller in ihrer Ausführung, wie die Ergebnisse des Vergleichs der verschiedenen Algorithmen aufzeigen (vgl. \autoref{subsec:resultsTools}).

Ungeachtet der schlechteren Ergebnisse erfordert die Anwendung des Random Testing-Ansatzes genauso wie der Ansatz des Combinatorial Testing eine umfassende Analyse des Testobjekts, insbesondere müssen abzuprüfende Bedingungen des Testsystems genauso analysiert und implementiert werden. Aus diesem Grund bietet der Random Testing-Ansatz nur in geringem Maße Vorteile in Bezug auf eine einfachere und effizientere Implementierung.

\item \textit{Inwieweit verändert die Einführung komplexerer Parameterstrukturen die Resultate im Vergleich zum einfacheren 'Basis-System'?}

Die Erweiterung des Basis-Systems zum Erweiterten System belegen, dass sich die Erkenntnisse der grundlegenden Anwendbarkeit von Combinatorial Testing im Anwendungsfall auf komplexere Problemstellungen skalieren lassen. 

Durch die Einführung zusätzlicher Parameter und zusätzlicher Bedingungen steigt die Komplexität des Testsystems an, was sich auch in den Resultaten aus \autoref{subsec:resultsComplexSystem} widerspiegelt: Über alle Interaktionsparameter hinweg wächst die Ausführungszeit, die Anzahl der abzudeckenden Werte-Kombinationen und die Anzahl der Testfälle beim Erweiterten System exponentiell an, während das Wachstum dieser Kenngrößen des Basis-Systems prinzipiell linear verläuft (vgl. \autoref{tab:resultsComplexSimple}, \autoref{fig:vergleichSimpleComplex}). Diese Erkenntnis entspricht den Erwartungen, welche aus \autoref{eq:proportionalität} über das Wachstumsverhalten der Anzahl der Testfälle hervorgeht.

Nichtsdestotrotz bleibt festzuhalten, dass die Erstellung der Testmengen via ACTS für das Erweiterte System eine geringe Ausführungszeit von weniger als einer Sekunde für alle Interaktionsparameter besitzt (vgl. \autoref{tab:resultsComplexSimple}). Somit lässt sich die Erstellung der Testmengen via Combinatorial Testing mit geringem Zusatzaufwand auf komplexere Systeme skalieren. Dabei gilt wie bereits für das Basis-System, dass die Analyse des Testobjekts und die Ermittlung passender Parameter, Werte und Bedingungen als größere Einflussfaktoren bei der Erstellung von Testmengen einzustufen sind als die Ausführung selbst.

\item \textit{Ist es sinnvoll spezifische Parameter zu extrahieren und für deren Werte jeweils unabhängig voneinander Testmengen mittels Combinatorial Testing zu erstellen?}

Letzterer Aspekt der vorherigen Teilfragestellungen spielt auch für die Beantwortung dieser Teilfragestellungen eine entscheidende Rolle: Das Extrahieren der Parameter Durchführungsweg / Art der Rückdeckung und die Erstellung von Testmengen für jede Kombination dieser beiden Parameter bewirkt eine Steigerung der Anzahl der Testfälle im Vergleich zum Erweiterten Systems (vgl. \autoref{subsec:resultsFullCombinations}). Diese fällt jedoch insbesondere für große Interaktionsparameter $t \in \{4,5,6\}$ gering aus, sodass die zusätzlichen Kosten der Erstellung von Testmengen via ACTS und der resultierenden Anzahl der Testfälle auch bei diesem Ansatz nicht in besonderem Maße ins Gewicht fallen.

Wie die Ausführungen aus \autoref{subsec:ImplFullCombinations} aufzeigen, ist der Ansatz der $t$-fachen Kombinatorik für jeden Durchführungsweg und jede Art der Rückdeckung besonders praxisrelevant und sinnvoll, da er die Zusammenhänge zwischen den verschiedenen Parametern von Easy Web berücksichtigt und etwaige Probleme einer naiven Anwendung der Methoden des Combinatorial Testing ausräumen kann. Die Umsetzung dieses Ansatzes erfordert jedoch ein tiefes Verständnis über die Zusammenhänge innerhalb von Easy Web und veranschaulicht die Relevanz der Erfahrung und Expertise der testenden Person bei der Anwendung der Methoden des Combinatorial Testing.

\item \textit{Welcher Algorithmus beziehungsweise welches Tool ist für die praktische Anwendung am nützlichsten?}

ACTS (vgl. \autoref{subsub:acts}) via IPOG (\autoref{subsub:ipog}) und PICT (vgl. \autoref{subsub:pict}) sind im Kontext der Ergebnisse dieser Arbeit als diejenigen Tools einzustufen, welche für die Praxisanwendung am nützlichsten erscheinen. Dies lässt sich einerseits auf die Resultate der Analyse der unterschiedlichen Algorithmen (vgl. \autoref{subsec:resultsTools}) und andererseits auf die Anwenderfreundlichkeit der Tools zurückführen.

In Bezug auf die Erstellung von Testmengen mit möglichst geringer Anzahl an Testfällen und geringer Ausführungszeit zeigt sich, dass IPOG und PICT über alle Interaktionsparameter hinweg eng beisammen liegen und geeignete Resultate liefern. In Bezug auf die Ausführungszeit ist IPOG via ACTS bei größeren Interaktionsparametern marginal schneller als PICT. 

Der schnellste Algorithmus der vier untersuchten Alternativen, IPOG-F (vgl. \autoref{subsub:ipog}), erstellt im Vergleich zu IPOG, PICT und dem vierten verglichenen Algorithmus CASA wesentlich größere Testmengen und ist daher in der praktischen Anwendung eher ungeeignet. Dieses Ergebnis widerspricht in gewisser Weise den erwarteten Ergebnissen, da IPOG-F laut seinen Autoren nicht nur in kürzerer Zeit, sondern auch in geringerer Anzahl Testfälle erstellen sollte (vgl. \autoref{subsub:ipog}). Eine mögliche Erklärung hierfür könnte sein, dass IPOG-F die Bedingungen des Basis-Systems ineffizienter in seinen Algorithmus integriert als die anderen Algorithmen.

Der CASA-Algorithmus generiert für $t \in \{2,3,4\}$ zwar die kleinste Menge an Testfällen für das Basis-System (vgl. \autoref{tab:resultsAlgorithmen}), besitzt jedoch einige Einschränkungen, welche die Anwendung in der Praxis erschweren. Als wesentlicher Aspekt lässt sich diesbezüglich die Volatilität von CASA anführen: Da CASA nicht deterministisch bei der Erstellung der Testmengen vorgeht, fallen die Ergebnisse des Algorithmus bei gleichbleibenden Rahmenbedingungen in verschiedenen Iterationen unterschiedlich aus und liefern nur in seltenen Fällen das optimale Minimum (vgl. \autoref{subsec:resultsTools}). Im konkreten Anwendungsfall wurde die minimale Anzahl der Testfälle des CASA-Algorithmus in lediglich zwei ($t=2$) und einer ($t \in \{3,4,5,6\}$) von 30 durchgeführten Iterationen ermittelt. Eine weitere Einschränkung des CASA-Algorithmus ergibt sich durch die erhöhte Ausführungszeit im Vergleich zu den anderen Algorithmen (vgl. \autoref{tab:resultsCASA}): Insbesondere für hohe Interaktionsparameter ($t \in \{4,5,6\}$) dauert die Ausführung des CASA-Algorithmus wesentlich länger als bei IPOG, IPOG-F und PICT.

Darüber hinaus besitzt CASA -- wie in \autoref{subsec:ImplAlgorithmen} beschrieben -- Probleme bei der Festsetzung einer oberen und unteren Schranke für die Anzahl der benötigten Testfälle bei Ausführung des Algorithmus: Im Speziellen geht CASA im Zusammenhang mit den zu berücksichtigenden Bedingungen des Basis-Systems von einer zu hohen Anzahl benötigter Testfälle aus und terminiert nicht immer. Aus diesem Grund erfordert die Anwendung von CASA die vorherige Ausführung von mindestens einem der beiden Algorithmen IPOG oder PICT, um sinnvolle Werte für die obere und untere Schranke des CASA-Algorithmus angeben zu können.

Auch die Anwenderfreundlichkeit von CASA bleibt hinter jener der beiden Tools ACTS und PICT zurück: Die Übersetzung der verschiedenen Parameter, deren Ausprägungen und Bedingungen in die interne Logik beziehungsweise Syntax von CASA fällt wesentlich komplexer aus als jene der Tools ACTS und PICT. CASA und PICT werden beide über die Kommandozeile gestartet und erfordern daher gewisse Grundkenntnisse im Umgang mit den Befehlen der Konsole. ACTS ermöglicht im Gegensatz dazu die einfachste Bedienmöglichkeit via Benutzeroberfläche, welche sich unter anderem dank einer ausführlichen Anleitung leicht nutzen lässt. Darüber hinaus besitzt ACTS zusätzlich die Möglichkeit über die Kommandozeile oder Java-API ausgeführt zu werden, was die vielfältige Nutzbarkeit des Tools unterstreicht.

\end{enumerate}

Anhand der verschiedenen Ansätze zur Beantwortung der allgemeinen Fragestellung dieser Arbeit und der Teilfragestellungen ergeben sich einige Limitierungen dieser Arbeit und mögliche Forschungsthemen für die Zukunft, die im Folgenden ausgeführt werden.

Im Allgemeinen beschränken sich die Ergebnisse dieser Arbeit auf einen einzigen Anwendungsfall mit einer reduzierten Anzahl an Parametern des verwendeten Testobjekts. Daher würde eine Ausweitung der verschiedenen Ansätze auf weitere Testobjekte beziehungsweise die Berücksichtigung zusätzlicher Parameter zur Generalisierung der Erkenntnisse beitragen.

Neben dieser grundlegenden Einschränkung der Arbeit existieren in Bezug auf die einzelnen Teilfragestellungen dieser Arbeit weitere, detaillierte Limitierungen und Erweiterungen:

\begin{itemize}
\item In Bezug auf den Vergleich zwischen Combinatorial Testing und Random Testing könnte insbesondere der Ansatz des zufallsbasierten Testens durch ein realitätsnahe Erweiterung ergänzt werden. In der praktischen Anwendung würde in den meisten Fällen keine derartig naive zufallsbasierte Testmethode zum Einsatz kommen, sondern vielmehr Methoden des Adaptive Random Testing (vgl. \autoref{subsub:randomTesting}) oder des erfahrungsbasierten Testens (vgl. \autoref{subsub:erfahrungsbasiertesTesten}). Dementsprechend wäre eine Gegenüberstellung von Combinatorial Testing mit diesen beiden Ansätzen für zukünftige Untersuchungen denkbar. Im speziellen Fall des erfahrungsbasierten Testens könnte man beispielsweise bestehende Testdaten aus der Praxis im Hinblick auf die kombinatorische Abdeckung überprüfen und im Zuge dessen eine mögliche Ersetzung durch Testfälle des Combinatorial Testing in Betracht ziehen.
\item Das Erweiterte System vereint im Vergleich zum Basis-System Veränderungen der Anzahl der Parameter, der Anzahl möglicher Ausprägungen je Parameter und der Anzahl der Bedingungen an das Testsystem: Dies führt dazu, dass sich Veränderungen der Resultate (vgl. \autoref{tab:resultsComplexSimple}) nicht eindeutig auf einen der drei Faktoren zurückführen lassen und der jeweilige Effekt der drei unterschiedlichen Faktoren unklar bleibt. Dies könnte durch eine detailliertere Herangehensweise ausgeräumt werden, indem jeweils zwei der drei Faktoren konstant gehalten werden, während der verbleibende Faktor verändert wird. Auch im Rahmen des Vergleichs der verschiedenen Algorithmen könnte eine derartige Herangehensweise hilfreich sein, um Unterschiede der Algorithmen im Umgang mit den Faktoren Anzahl der Parameter, Anzahl möglicher Ausprägungen je Parameter und Anzahl der Bedingungen herausarbeiten zu können.
\item Für die beiden Ansätze des Erweiterten Systems und der $t$-fachen Kombinatorik für jede/n Durchführungsweg / Art der Rückdeckung ergibt sich eine wesentliche Limitierung durch die Einschränkung des Testsystems auf wenige Parameter und eine daraus resultierende, reduzierte Kombinatorik. Insbesondere im Zusammenhang mit den Parametern Zuzahlung zu Beginn, Verwendeter Beitrag § 40 b und Verwendeter Beitrag § 3 Nr. 63 und deren Einschränkungen in Bezug auf mögliche Durchführungswege (vgl. \autoref{tab:beitragsgrenzen}) zeigt sich eine limitierte Aussagekraft der Ergebnisse: Beim Ansatz der $t$-fachen Kombinatorik für jede/n Durchführungsweg / Art der Rückdeckung wird bereits für $t \geq 3$ eine vollständige Kombinationsabdeckung für den Durchführungsweg Direktzusage erreicht, weshalb für höhere Interaktionsparameter keine weiteren Testfälle für die Direktzusage hinzukommen (vgl. \autoref{tab:ergebnisseFullCombination}). In ähnlicher Form lässt sich diese Erkenntnis auch auf die Ansätze des Basis-Systems und des Erweiterten Systems übertragen: Die allgemeine Aussagekraft der Resultate der verschiedenen Teilfragestellung wird dadurch in gewisser Weise eingeschränkt.
\end{itemize}