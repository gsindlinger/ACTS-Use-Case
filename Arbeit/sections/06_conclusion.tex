\chapter{Fazit}

Anhand des Anwendungsbeispiels der Beratungsplattform für Lebensversicherungsprodukten der ALH-Gruppe, Easy Web, wurde im Rahmen dieser Arbeit untersucht, inwiefern sich Methoden des Combinatorial Testing zur Erzeugung sinnvoller Testfälle anwenden lassen. Die Resultate zeigen auf, dass dies im Zusammenhang mit verschiedenen Parametern und Bedingungen des Testsystems generell möglich ist. Insbesondere im Vergleich zu naiven Methoden des zufallsbasierten Testens konnte dargelegt werden, dass Combinatorial Testing im Kontext des Anwendungsfalls erhebliche Vorteile bietet und zugleich wesentlich effizienter mit Bedingungen des Testobjekts umgehen kann als alternative, naive Herangehensweisen. Dies entspricht den Erkenntnissen verschiedener, aktueller Forschungsarbeiten.

Zwei untersuchte Teilfragestellungen konnten zudem belegen, dass sich die Anwendung von Combinatorial Testing unter geringfügigen, zusätzlichen Kosten skalieren lässt. Eine detaillierte Betrachtung des Einflusses der wesentlichen Einflussfaktoren auf die Skalierbarkeit der Methoden (Anzahl zu testender Parameter, Anzahl der zugehörigen Ausprägungen, Anzahl der Bedingungen des Testsystems) wurde in dieser Arbeit jedoch nicht vorgenommen und könnte Teil zukünftiger Forschung sein.

Die beiden Ansätze zur Prüfung der Skalierbarkeit dienten zudem als Methode zur Untersuchung der praktischen Nutzbarkeit von Combinatorial Testing: Als wesentliche Erkenntnis dieser Arbeit bleibt in diesem Zusammenhang festzuhalten, dass die adäquate Analyse des Testobjekts und Formulierung der Testparameter, ihrer Ausprägungen und der Bedingungen des Testsystems von essenzieller Bedeutung bei der Anwendung von Combinatorial Testing sind und fundierte Kenntnisse über das Testobjekt und das Testverfahren erfordern. Dies konnten existierende Forschungsarbeiten ebenfalls aufzeigen.

Darüber hinaus wurden in einer empirischen Analyse verschiedene Algorithmen und Tools zur Erstellung von Testmengen nach dem Prinzip des Combinatorial Testing verglichen: Es konnte dargelegt werden, dass die Algorithmen PICT und IPOG zuverlässig geringe Mengen an Testfällen bei kurzer Ausführungszeit erstellen und daher in der praktischen Anwendung gegenüber den anderen untersuchten Algorithmen CASA und IPOG-F zu bevorzugen sind. In Bezug auf die Nutzbarkeit stellte sich das Tool ACTS als nützlichstes Hilfsmittel bei der Erstellung von Testfällen heraus, unter anderem, da es als einziges der verglichenen Tools über eine grafische Benutzeroberfläche verfügt.




