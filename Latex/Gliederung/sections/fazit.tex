\chapter{Zusammenfassung}

Ausgehend von der Problemstellung, dass ein komplexes mathematisches Modell (vgl. Kapitel \ref{chap:modell}) m�glichst anschaulich visualisiert werden soll (vgl. Abschnitt \ref{sec:motivation}), wurde im Rahmen dieser Arbeit eine Anwendung entwickelt, welche diese Problematik unter Ber�cksichtigung allgemeiner Gestaltungsaspekte (vgl. Kapitel \ref{chap:visualisierungAllg}) aufl�sen soll. Dabei wurden die Konzepte einer mobilen Webanwendung (HTML, CSS, Javascript) verwendet, die im Gegensatz zu nativen Anwendungen den entscheidenden Vorteil haben, dass die entwickelte Anwendung �ber mehrere Betriebssysteme und Endger�te hinweg aufgerufen werden kann (vgl. Abschnitt \ref{sec:plattform}).

Die entwickelte Anwendung erm�glicht dem Nutzer �ber verschiedene Eingabefelder (vgl. Abschnitt \ref{sec:eingabefelder}) die Fragestellung zu beantworten, ob angesparte Mittel f�r die private Altersvorsorge bis zu einem vom Nutzer ausgew�hlten Alter reichen oder nicht. Die Beantwortung dieser Fragestellung erfolgt durch die mehrfache Simulation zuf�lliger Entnahmeprozesse anhand der ausgew�hlten Eingabeparameter des Nutzers: Die Wertentwicklung des w�hrend der Entnahme verbleibenden Kapitals wird mit dem Finanzmarktmodell von Black-Scholes beschrieben (vgl. Abschnitt \ref{subsec:black-scholes}). Drei verschiedene Darstellungen (vgl. Abschnitt \ref{sec:visualisierungen}) visualisieren die Ergebnisse der Simulationen und stellen den Bezug zur Ausgangsfragestellung her.

Der Vergleich mit bestehenden �hnlichen Konzepten (vgl. Abschnitt \ref{sec:vergleich}) zeigt, dass die entwickelte Anwendung sich in vielen Bereichen abheben kann. Insbesondere die Ber�cksichtigung der Schwankungen des Kapitalmarktes und die angemessene Visualisierung der Resultate im Hinblick allgemeiner Gestaltungsrichtlinien (vgl. Kapitel \ref{chap:visualisierungAllg}) sind die wesentlichen Aspekte in dieser Hinsicht. Als m�gliche Erweiterungen der entwickelten Anwendung (vgl. Abschnitt \ref{sec:limitierungen}) k�nnten zus�tzliche Eingabefelder, grunds�tzliche Erl�uterungen der Problemstellung, die Entwicklung einer hybriden beziehungsweise nativen Anwendung und die Optimierung der Leistungsf�higkeit der Anwendung in Betracht gezogen werden.

Die entwickelte Anwendung bietet daher eine weitreichende Hilfestellung in der Planung der Altersvorsorge: Auf anschauliche Art und Weise wird somit die Informationsbarriere zwischen den Anbietern und m�glichen Kunden dementsprechender Produkte der Altersvorsorge reduziert.